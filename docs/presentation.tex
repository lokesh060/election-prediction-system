\documentclass{beamer}
\usetheme{AnnArbor}

%gantt code
\usepackage{xcolor,colortbl}
\usepackage{graphicx}
\graphicspath{ {images/} }

% A package which allows simple repetition counts, and some useful commands

\usepackage{forloop}
\newcounter{loopcntr}
\newcommand{\rpt}[2][1]{%
  \forloop{loopcntr}{0}{\value{loopcntr}<#1}{#2}%
}
\newcommand{\on}[1][1]{
  \forloop{loopcntr}{0}{\value{loopcntr}<#1}{&\cellcolor{gray}}
}
\newcommand{\onx}[1][1]{
  \forloop{loopcntr}{0}{\value{loopcntr}<#1}{&\cellcolor{orange}}
}
\newcommand{\off}[1][1]{
  \forloop{loopcntr}{0}{\value{loopcntr}<#1}{&}
}

\definecolor{orange}{HTML}{FF7F00}

%end of gantt initialization code

\author[Naveen Vignesh.B, Silamabarasan.K]{
	 Naveen Vignesh.B -
	\textbf{110814104045}\\
	\and
	Silambarasan.K - 
	\textbf{110814104305}
	\newline \newline
	\textbf{Guided By:} G.Bharathi Mohan - Associate Professor, \\
	CSE Deptartment.
}
\institute[]{ \textbf{Jaya Engineering College}}
\title[Twitter Based Election Prediction]{ Election Results Prediction using Twitter Data}
\subtitle{By Applying Sentiment Analysis with Vector Modeling}
\date{\today}

\begin{document}
\titlepage
%abstract and objective
\begin{frame}{Abstract and Objective}
	\textbf{\textit{Abstract}} \newline
      To predict the election trends using Twitter Data and analyse the sentiment of people to give favourable prediction to the popular party.
	\newline \newline
	\textbf{\textit{Objectives}}	   
	\begin{itemize}
		\item To analyze the trend of people mindset.
		\item To give an qualitative prediction based on social data.
		\item To quantify the sentiment of people shown towards a particular political party.
		\item To analyse real time twitter data and give out the real time election wave.
	\end{itemize}
	
\end{frame}

%Litrature Survey
\begin{frame}{Literature Survey}
	\textbf{\textit{Predicting Election Results using Twitter and Linked Open Data}} \newline\
	\begin{itemize}
		\item A number of studies have recently explored Twitter's capabilities to predict outcomes
		\begin{enumerate}
			\item Elections [Tumasjan et al., 2010, Livne et al., 2011]
			\item Stock market [Bollen et al., 2011]
			\item Box office [Asur and Huberman., 2010]
		\end{enumerate}
	\item A recent studies revealed a number of small steps were involved to predict outcomes
	\item Each outcome affected the final result in a large scale.
	\end{itemize}
\end{frame}

%existing system
\begin{frame}{Existing System}
	\textbf{Manual Method}
	\begin{itemize}
		\item Go to physical location and take sample for local population.
		\item Find a cumulative average among N people to favour a particular party and give results.
	\end{itemize}
	\textbf{Polling using Social Media}
	\begin{itemize}
		\item Conducting polls in social media platforms like twitter to get opinion manually from user.
		\item User polling done without any filters like location or age or constituency eligibility.
		\item Allows duplicate submissions by users as aliases are not filtered.
	\end{itemize}
\end{frame}

%Proposed system
\begin{frame}{Proposed System}
	\begin{itemize}
		\item Performs predictions based on the nature of tweets related to particular election based on hashtags or user identifiers.
		\item Filteration of tweets based on nature of user with contrainst like age, location, country.
		\item Tweets are analysed on basis of sentiments of users expressed in the tweet by comparing with a dictionary sample - Positive and Negative set of Words as textual sample.
		\item Computing of favouratism based on predictor variables like
			\begin{enumerate}
				\item Number of tweets
				\item Timestamp of tweets
				\item User profile
			\end{enumerate}
		\item Calculating final result based on vector based probablistic quantifier and give out the prediction.
	\end{itemize}
\end{frame}

%advantage of proposed system
\begin{frame}{Advantage of Proposed System Over Existing System}
	\begin{itemize}
		\item Old method is active and requires the user to interact to give his/her poll opinion.
		\item New method is passive and doesn't need user interaction as it get suggestions from tweets.
		\item It is also more accurate as it analyses tweets from wide range of users.
	\end{itemize}
\end{frame}

%architectural designs
\begin{frame}{Architectural Design}
\begin{center}
	\includegraphics[width=18cm,height=9cm,keepaspectratio]{arch}
\end{center}
\end{frame}

%uml diagrams
\begin{frame}{Use Diagram}
\begin{center}
	\includegraphics[width=15cm,height=7cm,keepaspectratio]{usecase}
\end{center}

\end{frame}

\begin{frame}{Sequence Diagram}
\begin{center}
	\includegraphics[width=10cm,height=8cm,keepaspectratio]{sequence}
\end{center}
\end{frame}

\begin{frame}{Activity Diagram}
\begin{center}
	\includegraphics[width=18cm,height=6cm,keepaspectratio]{activity}
	\end{center}
\end{frame}


\begin{frame}{ER Diagram}
	\includegraphics[width=18cm,height=8cm,keepaspectratio]{er}
\end{frame}

%algorithms being used
\begin{frame}{Algorithms Used}
	\begin{itemize}
		\item Sentiment Analysis Using Dictionary Sample
		\item Vector based sentiment modeling
	\end{itemize}
\end{frame}
%modules an gantt chart
\begin{frame}{Modules and Gantt Chart}
\textbf{Modules}
\begin{enumerate}
	\item R-Script
	\item Twitter API Module
	\item GUI Module - Graph Projection
\end{enumerate}

\noindent\begin{tabular}{p{0.17\textwidth}
!{\vrule width 0.4mm}p{0.01\textwidth}*{3}{|p{0.01\textwidth}}
!{\vrule width 0.4mm}p{0.01\textwidth}*{3}{|p{0.01\textwidth}}
!{\vrule width 0.4mm}p{0.01\textwidth}*{3}{|p{0.01\textwidth}}
!{\vrule width 0.4mm}p{0.01\textwidth}*{3}{|p{0.01\textwidth}}
!{\vrule width 0.4mm}p{0.01\textwidth}*{3}{|p{0.01\textwidth}}
|}

\hline
% The top line
\textbf{Gantt chart} & \multicolumn{4}{c!{\vrule width 0.4mm}}{Month 1} 
           & \multicolumn{4}{c!{\vrule width 0.4mm}}{Month 2} 
           & \multicolumn{4}{c!{\vrule width 0.4mm}}{Month 3} 	
           & \multicolumn{4}{c|}{Month 4} 
      		\\
\hline
% The mbox prevent packages from being hyphenated
% The multicolumn produces no vertical guides within the columns it spans, but
% does put one at the end to complete the righ-hand edge of the table
Analysis   \on[3] \off[13] \\
\hline
Design  \off[3] \on[4] \off[9] \\
\hline
% Note the omitting the count to on or off is the same as setting the count to 1
Coding and Testing \off[6] \onx[8] \off[2] \\
\hline
Testing and Deployment \off[11] \on[5] \\
\hline
\end{tabular}

\end{frame}

%Expected Outcomes
\begin{frame}{Expected Outcomes}
	\begin{itemize}
		\item The solution will perform passive analyis real-time twitter data without disturbing the users.
		\item The solution proposed will be standalone application running on it own server thread analysing twitter data realtime for N days before election day.
		\item The software will give overall rating for each day for respective party and will find cumulative distributive result of the overall standings.
		\item The solution will be an R-Script that can be deployed even on a R-Server to monitor the real time scenario in social media.
	\end{itemize}
\end{frame}

%Hardware and software requirements
\begin{frame}{Hardware and Software Requirements}
	\textbf{\textit{Software Tools Used:}}
	\begin{itemize}
		\item R Framework 3.4.1 - Community Edition
		\item RStudio 1.0.143 - Community  Edition
		\item Twitter API
	\end{itemize} 

	\textbf{\textit{Minimum Hardware Requirements:}}
	\begin{itemize}
		\item Intel-Compatible platform running Windows XP/Vista/7/8/10.
		\item 32 MB of RAM.
		\item 200 MB of HDD space.
		\item Internet Connectivity.
	\end{itemize}
\end{frame}

%reference
\begin{frame}{Reference}
	\begin{itemize}
		\item Kalampokis, Karamanou, Tambouris, Tarabanis - On Predicting Election Results using Twitter and Linked Open Data - \textbf{Journal of Universal Computer Science, vol. 23, no. 3 (2017), 280-303}
	 	\item Ausr, S.Huberman, B.A. (2010) - Predicting the future with social media - \textbf{Proceedings of 2010 IEEE/WIC/ACM International Conference.}
	 	\item Bollen, J., Mao, H., Zeng, X. - Twitter mood predicts the stock market - \textbf{Journal of Computational Science}
	 	\item Tumasjan, A., Sprenger, T.,Sandner, P.,Welpe, - Predicting elections with twitter: What 140 characters reveal about political sentiment - \textbf{4th International AAA1 Conference on Weblogs and Social Media}
	\end{itemize}
\end{frame}

\end{document}